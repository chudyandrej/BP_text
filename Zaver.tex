
\chapter{Záver}
Cieľom tejto práce bolo vytvoriť aplikáciu, ktorá bude schopná počítať prechody osob s čo najväčšou presnosťou a zachovania priaznivej ceny. Vrámci práce boli vytvorené dve aplikácie. Jedna pre snímanie priestoru virtuálnej brány za pomoci 2D technológie a druhá, ktorá používa pre snímanie aktívne hĺbkomery (príloha \ref{pr:CD}).

V prípade aplikácie využívajúcej 2D snímanie sa podarilo vytvoriť nasaditeľnú aplikáciu, ktorej úspešnosť detekcie sa pohybuje medzi \textbf{90 - 94\%}. Cenu takéhoto zariadenia sa podarilo stlačiť na \textbf{70 EUR} vďaka použitiu malého a lacného počítača Rraspberry Pi a štandardnej širokouhlej webkamery. Nízka cena zariadenia dovoľuje použiť takýto produkt pre rôzne maloobchodné prevádzky, ktoré potrebujú informácie o počte návštev zo štatistických dôvodov a nižšia spoľahlivosť zariadenia im neprekáža. Takéto zariadenia majú potenciál nahradiť nepresné IR závory, ktoré sú stále veľmi často používané kvôli cenovej nedostupnosti presnejších riešení.

V prípade druhej aplikácie, ktorá detekuje prechod osôb za pomocí hĺbkového snímku priestoru brány, bolo na začiatok nutné vybrať najvhodnejší hĺbkomer. Vrámci práce boli k dispozícii žiaľ len tri a všetky používali technológiu aktívnej triangulácie (sekcia~\ref{sec:activeDeep}). Napriek tomu sa podarilo vytvoriť systém pre porovnanie kvality hĺbkového snímku na základe priemerného počtu chybne nameraných pixlov. Práca ukázala, že Kinect 360 má väčšiu odolnosť voči interferencii denného svetla avšak nie je možné určiť, ktorý hĺbkový snímač je pre aplikáciu najlepší. Je to z dôvodu rozdielnych zorných polí kamier. Preto na výber najvhodnejšej kamery je nutné poznať presné miesto nasadenia systému. Nameraná úspešnosť aplikácie vytvorenej vrámci tejto práce bola \textbf{96 - 98\%}. Cena sa pohybuje okolo \textbf{150 - 220 EUR} v závislosti na type použitého hĺbkomera, keďže predstavuje najdrahšiu položku celého produktu. Vďaka vysokej presnosti je tento typ aplikácie vhodný na príklad v aplikáciach objektovej bezpečnosti v spojení s ďalším systémom overujúcim identitu prechádzaného objektu alebo v aplikáciach, kde je potrebné zaznámenávať väčšie množstvo informácií, ako napríklad výšku človeka.

