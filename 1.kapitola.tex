%=========================================================================
% (c) Michal Bidlo, Bohuslav Křena, 2008

\chapter{Úvod}
Väčšina systémov používaná pre zaznamenávanie počtu osôb prechádzajúcich virtuálnymi bránami  je založená na technológií infračervenej závory, z dôvodu nízkej ceny. Tento systém je však veľmi nepresný, pretože nedokáže detekovať viac ľudí prechádzajúcich vedľa seba alebo oproti sebe, čo je veľmi častá situácia, ktorá zapríčiňuje značnú mieru chybovosti. Pri použití iných systémov je nutné prenášať gigabajty dát po sieti alebo sú veľmi drahé (stovky EUR). Preto je nutné vytvoriť systém, ktorý bude presnejší ale zároveň si zachová priaznivú cenu. Touto problematikou sa zaoberá táto práca. Spojením lacnej webkamery a nízkonákladového počítača Raspberry Pi je možné zrealizovať systém s vysokou mierou presnosti a veľmi priaznivou cenou.
    
Informácie vytvorené takýmto systémom môžu byť veľmi užitočné, napríklad pre obchodné reťazce, ktoré na základe dlhodobej štatistiky návštevnosti jednotlivých predajní dokážu pripraviť obchodné stratégie, vyhodnotiť dopady reklamnej kampane na návštevnosť či vyrátať úspešnosť predaja zamestnancov ako pomer zákazníkov a predaného tovaru. Všetky tieto dáta sú kľúčovým faktorom pre analýzu obchodných stratégii a ich využitie vedie k zefektívneniu prevádzky.
    
Ďalšie využitie takéhoto systému je na miestach s obmedzenou kapacitou ako napríklad kúpaliská, štadióny, divadlá alebo sklady, kde na základe informácie o počte zamestnancov na jednotlivých oddeleniach, je možné dynamicky prerozdeľovať pracovnú silu v závislosti na množstve požiadavkou na jednotlivé oddelenia. 



\section{Existujúce riešenia}
Problematikou počítania (detekovania) prechodov ľudí sa zaoberá celá rada aplikácií.  
Jedna zo základných aplikácií je IR závora, ktorú bežne môžeme nájsť na akejkoľvek garážovej bráne. Je založená na princípe vysielača a prijímača. Vysielač vysiela infračervený lúč a keď niekto prejde, lúč sa zatieni a rele v prijímači sa zopne. Táto technológia má však veľmi veľa problémov a obmedzení. Najväčším obmedzením je, že môže byť inštalovaná len v zúžených priestoroch, kde môže v jednom okamihu prechádzať len jeden človek. Toto tvrdenie sa opiera o fakt, že ak by išli dvaja alebo viacerí ľudia vedľa seba, senzor to zaznamená ako jedno zopnutie a teda prechod len jednej osoby.

\vspace{8mm}
 
Firma riSys\footnote{Oficiálne zastúpenie spoločnosti iriSys na slovensku: \url{http://www.irisys.sk/people-counting}} ponúkajú vo svojom portfóliu aplikačné riešenie problému počítania osôb. Využívajú na to termálne kamery. Výrobca uvádza, že ich produkt dosahuje úspešnosť na úrovni 98\%. Avšak cena za najlacnejší model sa pohybuje okolo 1000 EUR za kus. Táto firma na svojich stránkach ďalej uvádza, že infračervená technológia, ktorú používajú je v dnešnej dobe najpresnejšou metódou počítania prechodov osôb na trhu. 

Ďalšou spoločnosťou, ktorá sa venuje problematike vytvárania štatistických dát z počítania prechodu osôb, odhad veku či spokojnosti zákazníkov je firma Brickstream\footnote{Oficiálna stránka spoločnosti: \url{http://www.brickstream.com}}. Vytvorili malé zariadenie, ktoré na základe stereoskopickej technológie dokáže zaznamenávať jednotlivé transakcie (prechody). Výrobca uvádza presnosť na úrovni 97\%.


Ostatné riešenia sa poväčšine opierajú o technológiu obyčajnej kamery. Tento prístup je veľmi lacný, lebo kamera nemusí byť ničím výnimočná dokonca množstvo objektov má kamery už predom inštalované. Postupom času rôzni výrobcovia, začali dokonca implementovať takéto technológie do firmwarov kamier za pomoci hardvérovej akcelerácie. Dokážu počítať ľudí, spustiť alarm pri narušení virtuálnej zóny či čítať poznávacie značky aut. Pri testoch však bolo dokázané, že presnosť takýchto služieb zo strany kamerových spoločností je skôr podpriemerná. 





